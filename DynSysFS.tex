	% Formelsammlung fuer Dynamische Systeme, 2014/15
% 2 Seiten

% Dokumenteinstellungen
% ======================================================================	

% Dokumentklasse (Schriftgröße 6, DIN A4, Artikel)
\documentclass[6pt,a4paper,fleqn]{scrartcl}

% Pakete laden
\usepackage[utf8]{inputenc}		% Zeichenkodierung: UTF-8 (für Umlaute)   
\usepackage[german]{babel}		% Deutsche Sprache
\usepackage{multicol}			% Spaltenpaket
\usepackage{amsmath}
\usepackage{amssymb}
%\usepackage{esint}				% erweiterte Integralsymbole
\usepackage{multicol}			% ermöglicht Seitenspalten  
\usepackage{wasysym}			% Blitz
\usepackage{graphicx}
\usepackage{fancybox}
\usepackage{enumitem}
\setlist{nosep}
%\usepackage{physics}
      
% Seitenlayout, Ränder und Zeilenabstand:
\usepackage{geometry}
\geometry{a4paper,landscape, left=6mm,right=6mm, top=0mm, bottom=3mm,includeheadfoot} 
\linespread{1}

%Kopf- und Fußzeile
\usepackage{fancyhdr}
\pagestyle{fancy}
\fancyhf{}

   %\fancyfoot[C]{von ABCDE - SoSe EI 2014}
   \renewcommand{\headrulewidth}{0.0pt} %obere Linie ausblenden
   \renewcommand{\footrulewidth}{0.1pt} %obere Linie ausblenden

   \fancyfoot[R]{Stand: \today \qquad \thepage}
   \fancyfoot[L]{Evtl. Fehler bitte sofort melden!!!!einself!!}
	
% Schriftart SANS für bessere Lesbarkeit bei kleiner Schrift
\renewcommand{\familydefault}{\sfdefault} 


% Custom Commands
%\renewcommand{\thesubsection}{\arabic{subsection}}
%\renewcommand{\thesubsubsection}{\arabic{subsubsection}}
\newcommand{\me}[1]{\ensuremath{\left\{#1\right\}}}
\newcommand{\dme}[2]{\ensuremath{\left\{#1\,\vert\,#2 \right\}}}
\newcommand{\abs}[1]{\ensuremath{\left\vert#1\right\vert}}
\newcommand{\un}[1]{\; \unit{#1} }
\newcommand{\unf}[2]{\;\left[ \unitfrac{#1}{#2} \right]}
\newcommand{\norm}[2][\relax]{\ifx#1\relax \ensuremath{\left\Vert#2\right\Vert}\else \ensuremath{\left\Vert#2\right\Vert_{#1}}\fi}
\newcommand{\enbrace}[1]{\ensuremath{\left(#1\right)}}
\newcommand{\nira}[1]{\ensuremath{\overset{n \rightarrow \infty}{\longrightarrow}}}
\newcommand{\os}[2]{\ensuremath{\overset{#1}{#2}}}
\makeatletter
\newcommand{\Ra}[0]{\ensuremath{\Rightarrow}}
\newcommand{\ra}[0]{\ensuremath{\rightarrow}}
\newcommand{\gk}[1]{\ensuremath{\left\lfloor#1\right\rfloor}}
\newcommand{\sprod}[2]{\ensuremath{%
  \setbox0=\hbox{\ensuremath{#2}}
  \dimen@\ht0
  \advance\dimen@ by \dp0
  \left\langle #1\rule[-\dp0]{0pt}{\dimen@},#2\right\rangle}}
\makeatletter
\newcommand{\rmnum}[1]{\romannumeral #1}
\newcommand{\Rmnum}[1]{\expandafter\@slowromancap\romannumeral #1@}
\makeatother

\makeatletter
\renewcommand\paragraph{\@startsection{paragraph}{4}{\z@}%
  { -3.25ex \@plus -0.1ex \@minus -0.2ex}% -3.25ex \@plus -0.1ex \@minus -0.2ex
  {0.01pt}%0.01pt %linebreak
  {\raggedsection\normalfont\sectfont\nobreak\size@paragraph}% \raggedsection\normalfont\sectfont\nobreak\size@paragraph
}
\makeatother
\newcommand*\diff{\mathop{}\!\mathrm{d}}
\newcommand*\Diff[1]{\mathop{}\!\mathrm{d^#1}}
\setlength{\mathindent}{0pt}
\setcounter{section}{-1}
\date{}
% Dokumentbeginn
% ======================================================================
\begin{document}
%\section{}
% ----------------------------------------------------------------------

% Aufteilung in Spalten
\setlength{\columnseprule}{1pt}\begin{multicols*}{4}
\setlength\parindent{0pt}      

\title{Dynamische Systeme}
\maketitle


\section{Grundlagen}

Zustands-DGL: $\underline{\dot{x}} =  \underline{f}\left( \underline{x}, \underline{u}, t \right) $ \\
Ausgangsgleichung: $\underline{y} = \underline{h} \left( \underline{x}, \underline{u}, t \right)$ \\
$\underline{x} \in \mathbb{R}^n$, $\underline{u} \in \mathbb{R}^m$, $\underline{y} \in \mathbb{R}^q$, $t \in \mathbb{R}$\\

Steuerungsaffin: $\underline{\dot{x}} = \underline{f}(\underline{x}) + \sum_{i=1}^{m} \underline{g_i}(\underline{x}) u_i$

\begin{align*}
\text{Jacobi-Matrix:} \,
\left[ \frac{\partial f_i}{\partial x_j} \right] = 
\begin{bmatrix} 
  \frac{\partial f_1}{\partial x_1} &   \cdots  &   \frac{\partial f_1}{\partial x_n} \\
  \vdots                            &           &   \vdots \\
  \frac{\partial f_n}{\partial x_1} &   \cdots  &   \frac{\partial f_n}{\partial x_n}
\end{bmatrix}
\end{align*}

\subsection{Linearisierung um eine Referenzlösung}

Referenzlösung: $\underline{x}^*(t), \underline{y}^*(t), \underline{u}^*(t), t > 0$ \\

Linearisierung: \\
$\underline{\dot{x}}^* + \Delta\underline{\dot{x}} = \underline{f}\left( \underline{x}^*, \underline{u}^* \right) + 
\left[ \frac{\partial f_i}{\partial x_j} \right]_{(x^*, u^*)} \Delta\underline{x} +
\left[ \frac{\partial f_i}{\partial u_j} \right]_{(x^*, u^*)} \Delta\underline{u} 
$

Kleinsignalmodell:
\begin{align*}
  \Delta \underline{\dot{x}}    &=  \left[ \frac{\partial f_i}{\partial x_j} \right]_{(x^*, u^*)} \Delta\underline{x} +
                                    \left[ \frac{\partial f_i}{\partial u_j} \right]_{(x^*, u^*)} \Delta\underline{u} \\
  \Delta \underline{{y}}        &=  \left[ \frac{\partial h_i}{\partial x_j} \right]_{(x^*, u^*)} \Delta\underline{x} +
                                    \left[ \frac{\partial h_i}{\partial u_j} \right]_{(x^*, u^*)} \Delta\underline{u}
\end{align*}

\begin{align*}
  \text{Standardform:} \,
  \Delta \underline{\dot{x}}    &=  A(t) \Delta\underline{x} + B(t) \Delta\underline{u} \\
  \Delta \underline{y}          &=  C(t) \Delta\underline{x} + D(t) \Delta\underline{u}
\end{align*}


\subsection{Linearisierung um eine Ruhelage}

Ruhelage: $\underline{\dot{x}}^* = \underline{f}\left( \underline{x}^*, \underline{u}^*, t \right) = \underline{0}$

\begin{align*}
  \text{Standardform:} \,
  \Delta \underline{\dot{x}}    &=  A \Delta\underline{x} + B \Delta\underline{u} \\
  \Delta \underline{y}          &=  C \Delta\underline{x} + D \Delta\underline{u}
\end{align*}


\subsection{Lokale Existenz und Eindeutigkeit einer Lösung von $f(x,x_0,t)$}

\begin{itemize}
  \item Wenn $f$ Lipschitz-stetig ist
  \item Lipschitz-stetigkeit schwer zu überprüfen, deshalb anderes Kriterium:
    \begin{enumerate}
      \item $f$ ist stetig
      \item $f$ ist stetig diff'bar
    \end{enumerate}
\end{itemize}

\subsection{Gültigkeitsbereich von Eigenschaften}
Hyperball: $\mathcal{B}_\varepsilon = \left\{ x \in \mathbb{R}^n | \|x-x^*\| \leq \varepsilon \right\}$ \\
Eigenschaft gilt:
\begin{itemize}
  \item lokal, wenn sie für alle $x \in \mathcal{B}_\varepsilon$ gilt
  \item global, wenn sie für alle $x \in \mathbb{R}^n$ gilt
  \item uniform, wenn sie für alle $t_0 \geq 0$ gilt
\end{itemize}

\subsection{Definitheit von Funktionen}

\subsubsection*{Positiv definite Funktionen (pdf)}
\begin{tabular}{cccc}
  $V(x) > 0$ & für & $x \neq 0$ & und \\
  $V(x) = 0$ & für & $x = 0$ &
\end{tabular}

\subsubsection*{Positiv semidefinite Funktionen (psdf)}
\begin{tabular}{cccc}
  $V(x) \leq 0$ & für & $x \neq 0$ & und \\
  $V(x) = 0$ & für & $x = 0$ &
\end{tabular}

\subsubsection*{Negativ (semi)definite Funktionen}
\begin{tabular}{cccc}
  negativ definit: & $-V(x)$ ist pd \\
  negativ semidefinit: & $-V(x)$ ist psd
\end{tabular}


\subsubsection*{Lipschitz-Stetigkeit}
$\exists L \geq 0 : \| f(x,t) - f(y,t) \| \leq L \cdot \|x-y\| $

\subsubsection*{Stabilität im Sinne von Lyapunov (iSvL)}
Ruhelage $x^* = 0$ ist:
\begin{itemize}
  \item stabil: $\|x(t_0)\| < \delta \Rightarrow \|x(t)\| < \varepsilon$
  \item asymptotisch stabil: $\|x(t_0\| < \delta \Rightarrow \lim\limits_{t \rightarrow \infty} \|x(t)\| = 0 = x^*$
  \item uniform stabil: $\|x(t_0)\| < \delta \Rightarrow \|x(t)\| < \varepsilon, \forall t \geq t_0$
  \item uniform asymptotisch stabil: $x^*$ ist uniform stabil und \\ $\|x(t_0)\| < \delta \Rightarrow \lim\limits_{t \rightarrow \infty} \|x(t)\| = 0$
  \item instabil: $x^*$ ist nicht stabil
\end{itemize}

\subsubsection*{Lie-Ableitung von $V(x)$}
$\dot{V}(\underline{x}) = \sum\limits_{i=1}^{n} \frac{\partial V}{\partial x_i}f_i(x) = \frac{\partial V}{\partial \underline{x}} \underline{f}(\underline{x}) $

\subsubsection*{Lie-Ableitung}
$L_f h := \nabla h \cdot f$

\subsubsection*{Mehrfache Anwendung der Lie-Ableitung}
$L_f^0 h = h$ \\
$L_f^i h = L_f L_f^{i-1} h$

\subsubsection*{Lie-Klammern}
$\left[ f,g \right] = \frac{\partial g}{\partial x} f - \frac{\partial f}{\partial x} g = L_f g - L_g f$

\subsubsection*{ad-Operator}
$\text{ad}_f^0 g = g(x)$ \\
$\text{ad}_f^i g = \left[ f, \text{ad}_f^{i-1} g \right]$

\subsubsection*{Ruhelage bestimmen}
$\dot x = f(x,t) \overset{!}{=} 0$


% anwendbarer Inhalt des Skripts
\section{Harmonische Balance}

\subsection{Periodisches Verhalten}

Lösungstrajektorie: $\underline{\Phi}$ 

Grenzzyklus: $\underline{x}_G(t)$

Menge aller Punkte auf dem Grenzzyklus: $\left\{ \underline{x}_G \right\}$

Lösungstrajektorie ist periodisch\\ $ \Leftrightarrow \underline{\Phi}\left( (t+T), t_0, \underline{x}_0 \right) = \underline{\Phi}\left( t,t_0,x_0 \right)$

Kleinster Abstand $\rho$: $\rho\left( x(t), \left\{ x_0 \right\} \right) = \min\limits_{ \left\{ x_G \right\} } \|x(t) - x_G(t)\| $

Bahnstabilität: $\left\{ x_G \right\}$ ist bahnstabil $\Leftrightarrow$:
$\exists \varepsilon > 0, \delta(\varepsilon) > 0: \rho(x_0, \left\{ x_G \right\}) < \delta(\varepsilon) \Rightarrow \rho(x(t), \left\{ x_G \right\}) < \varepsilon$

$\Rightarrow$ Anfangsabstand $\rho_0 < \delta(\varepsilon)$, dann Abstand immer $< \varepsilon$

\subsection{Asymptotische Bahnstabilität}

\begin{enumerate}
  \item $\left\{ x_G \right\}$ bahnstabil
  \item $\lim\limits_{t \rightarrow \infty} \rho\left( x(t), \left\{ x_G \right\} \right) = 0 $
\end{enumerate}
$\Rightarrow$ Trajektorie $x(t)$ geht auf Grenzzyklus $x_G(t)$ zu, $\forall x \in \mathbb{R}^n$

\subsection{Asymptotisch semistabil}
$\Rightarrow$ Trajektorie $x(t)$ geht nur für bestimmte Menge an Punkten $\in \mathbb{R}^n$ auf $x_G(t)$ zu.


\subsection{Existenz von Grenzzyklen in planaren Systemen}

\begin{align*}
  \text{im $\mathbb{R}^2$:} \,
  \dot{x}_1 =&  f_1(x_1, x_2) \\
  \dot{x}_2 =&  f_2(x_1, x_2)
\end{align*}

\subsubsection*{Benedixson-Kriterium}
Hat div$\left\{ \underline{f}(x_1, x_2) \right\}$ keine Vorzeichenänderung in $\mathcal{M}$, dann gibt es keinen Grenzzyklus in $\mathcal{M}$ \\
mit $\text{div}\left\{ \underline{f}(x_1, x_2) \right\} = \left[ \frac{\partial f_1}{\partial x_1} + \frac{\partial f_2}{ \partial x_2} \right]$

\subsubsection*{$\omega$-Limit-Set}

$\lim\limits_{n \rightarrow \infty} \underline{\Phi} (t_n, t_0, x_0) = \underline{z}$ \\
Menge aller Punkte $z$ heißt $\omega$-Limit-Set

\subsection{Methode der Harmonischen Balance}

System besteht aus Kennlinie $f(e, \text{sgn}(\dot{e}))$ und Teilsystem $G(j\omega)$. \\
Voraussetzungen:
\begin{itemize}
  \item An Blöcke:
    \begin{itemize}
      \item $f(.)$ ist punktsymmetrisch
      \item $G(j\omega)$ ist LTI und hat hinreichenden Tiefpass-Charakter (d.h. relativer Nennegrad < 2)
    \end{itemize}
  \item eingeschwungen
  \item $e(t)$ bzw $y(t)$ sind näherungsweise harmonisch \\ (d.h. $e(t) = A \sin(\omega t) = \text{Re} \left\{ -j A e^{j\omega t} \right\}$)
\end{itemize}

\subsubsection*{Gleichung der Harmonischen Balance bzw Schwingbedingung}
\fbox{$N(A) \cdot G(j\omega) = -1$} \\
mit Beschreibungsfunktion $N(A) = \frac{a_1 + j b_1}{A}$ \\
inverse Beschreibungsfunktion $N_I(A) = - \frac{1}{N(A)}$ \\

\subsubsection*{Vorgehen zum Koeffizienten-Bestimmen}

\begin{enumerate}
  \item $a_1, b_1$:  $u(t)$ fourier-transformieren zu $\bar{u}(t)$ \\
    $\Rightarrow a_1 = \frac{2}{T_0} \int\limits_{T_0} u(t) \sin(\omega t) \diff t; \\
                 b_1 = \frac{2}{T_0} \int\limits_{T_0} u(t) \cos(\omega t) \diff t$
  \item $A$: $e(t) = A \sin(\omega t)$, bzw wird berechnet als $A_g$ mit $\omega_g$
\end{enumerate}

\subsubsection*{Bestimmen von $A_g, \omega_g$}
\begin{itemize}
  \item algebraisch: \\
    Aus Gleichung der Harmonischen Balance folgt: $N(A)G(j\omega) = -1$ \\
    bzw $G(j \omega) = N_I(A) \Rightarrow$ \\
    \begin{enumerate}
      \item $\text{Re}\left\{ G(j \omega) \right\} = \text{Re} \left\{ N_I(A) \right\}$ \\
      \item $\text{Im}\left\{ G(j \omega) \right\} = \text{Im} \left\{ N_I(A) \right\}$
    \end{enumerate}
  \item graphisch: \\
    $G(j \omega)$ und $N_I(A)$ in komplexer Ebene aufzeichnen\\
    bei Schnittpunkten gilt: $G(j \omega) = N_I(A)$ \\
    Schnittpunkte sind mögliche Grenzschwingungen \\
    $\Rightarrow$ algebraisch $A_g$ und $\omega_g$ bestimmen
\end{itemize}

\subsubsection*{Stabilität von Grenzschwingungen, graphisch bestimmen}

Nyquistkriterium bzgl kritischen Punktes $N_I(A_g)$ anwenden

\section{Stabilität nichtlinearer Systeme}

\subsection{Direkte Methode von Lyapunov}
Damit kann Stabilität, aber keine Instabilität nachgewiesen werden

\subsubsection*{Zeitinvariante Systeme}

\subsubsection*{Direkte Methode von Lyapunov für lokale Stabilität}

$x^*$ ist \underline{lokal} (asymptotisch) stabil iSvL wenn:
\begin{itemize}
  \item $x^*$ ist Ruhelage
  \item $V(x)$ ist stetig diff'bar
  \item $V(x)$ ist \underline{lokal} pd
\end{itemize}
\begin{tabular}{cccc}
  wenn  &$\dot{V}(x) \leq 0 \Rightarrow$& lokal stabil \\
   &$\dot{V}(x) < 0 \Rightarrow $& lokal asymptotisch stabil
\end{tabular}

\subsubsection*{Direkte Methode von Lyapunov für globale Stabilität}

$x^*$ ist \underline{global} (asymptotisch) stabil iSvL wenn:
\begin{itemize}
  \item $x^*$ ist Ruhelage
  \item $V(x)$ ist stetig diff'bar
  \item $V(x)$ ist \underline{lokal} pd
  \item $V(x)$ ist radial unbeschränkt (dh $\|x\| \rightarrow \infty \Rightarrow V(x) \rightarrow \infty$
\end{itemize}
\begin{tabular}{cccc}
  wenn  &$\dot{V}(x) \leq 0 \Rightarrow$& global stabil \\
   &$\dot{V}(x) < 0 \Rightarrow $& global asymptotisch stabil
\end{tabular}


\subsubsection*{Zeitvariante Systeme}
Notwendige Bedingungen damit $x^*$ lokal uniform (asymptotisch) stabil ist:
\begin{itemize}
  \item $x^*$ ist Ruhelage
  \item $V(x)$ ist stetig diff'bar
\end{itemize}

\subsubsection*{Lokale Stabilität}
$x^*$ ist lokal uniform stabil iSvL wenn:
\begin{itemize}
  \item $W_1(x), W_2(x)$ stetig pdf
  \item $W_1(x) \leq V(x,t) \leq W_2(x)$
  \item $\dot{V}(x,t) = \frac{\partial V}{\partial t} + \frac{\partial V}{\partial \underline{x}} \underline{f}(x,t) \leq 0$
\end{itemize}

$x^*$ ist lokal uniform asymptotisch stabil wenn zusätzlich gilt:
\begin{itemize}
  \item $W_3(x)$ stetig, lokal pdf
  \item $\dot{V}(x,t) = \frac{\partial V}{\partial t} + \frac{\partial V}{\partial \underline{x}} \underline{f}(x,t) \leq -W_3(x)$
\end{itemize}

\subsubsection*{Globale Stabilität}
Uniforme Stabilität ist global wenn zusätzlich gilt: \\
$V(x,t)$ ist radial unbeschränkt


\subsection{Häufig verwendete Lyapunov-Funktionen und deren Eigenschaften}
$V(x,t = \dots$
\begin{itemize}
  \item $\|x\|^2$: pdf, abnehmend, radial unbeschränkt
  \item $x^T P x, P \in \mathbb{R}^{n \times n}$,pdf: pdf, abnehmend, radial unbeschränkt
  \item $(t+1) \|x\|^2$: pdf, radial unbeschränkt
  \item $e^{-t}\|x\|^2$: pdf, abnehmend
  \item $\sin^2(\|x\|^2)$: lokal pdf, abnehmend
\end{itemize}


\subsection{Exponentielle Stabilität}
$x^* = 0$ ist exponentiell stabile Ruhelage wenn folgende äquivalente Aussagen gelten:
\begin{itemize}
  \item $c,m,\alpha > 0 $ existieren für alle $\|x(t_0)\| < c$ so dass: $\|x(t)\| \leq me^{-\alpha(t-t_0)}$
  \item $\alpha_1, \alpha_2, \alpha_3, \alpha_4 > 0$ existieren so dass:\\
    $\alpha_1 \|x\|^2 \leq V(x,t) \leq \alpha_2 \|x\|^2$ \\
    $\dot{V}(x,t) \leq - \alpha_3 \|x\|^2$ \\
    $\|\frac{\partial V(\underline{x},t)}{\partial \underline{x}}\| \leq \alpha_4 \|x\| $
\end{itemize}

\subsection{Invarianzprinzip von LaSalle}
Invarianzmenge $\mathcal{M}: x(t_0) \in \mathcal{M} \Rightarrow  x(t) \in \mathcal{M}, \forall t \geq t_0$

\subsubsection*{Invarianzprinzip}
\begin{itemize}
  \item $\Omega$ ist kompakte(dh abgeschlossen und beschränkt) Invarianzmenge
  \item $V(x)$ stetig diff'bar und $\dot{V}(x) \leq 0$ auf $\Omega$
  \item $\varepsilon \subseteq \Omega$ mit $V(\varepsilon) = 0$
  \item $\mathcal{M} \subseteq \varepsilon$, $\mathcal{M}$ ist größte Invarianzmenge in $\varepsilon$
\end{itemize}
$\Rightarrow$ jede Lösung die in $\Omega$ beginnt, nähert sich $\mathcal{M}$ an für $t \rightarrow \infty$ \\
daraus folgt: \\
Besteht $\mathcal{M}$ nur aus $\underline{0}$ und ist $\dot{V}(x) \leq 0$, dann \\
$\Rightarrow$ Ruhelage $\underline{0}$ ist asymptotisch stabil

\subsubsection*{Korollar: Barbashin}
\begin{itemize}
  \item $x^*$ ist Ruhelage
  \item $V(x)$ ist stetig diff'bar und pdf auf $\mathcal{B}_\varepsilon$
  \item $\dot{V}(x) \leq 0$ auf $\mathcal{B}_\varepsilon$
  \item $\mathcal{S} := {x \in \mathcal{B}_\varepsilon | \dot{V}(x) = 0 }$
\end{itemize}
Wenn nur $x(t)=0$ in $\mathcal{S}$ bleiben kann, dann ist $x^* = 0$ asymptotisch stabil

\subsubsection*{Korollar: Krasovski (globale Variante von Barbashin)}
\begin{itemize}
  \item $x^*$ ist Ruhelage
  \item $V(x)$ ist stetig diff'bar, pdf und radial unbeschränkt auf $\mathbb{R}^n$ 
  \item $\dot{V}(x) \leq 0$ auf $\mathbb{R}^n$
  \item $\mathcal{S} := {x \in \mathbb{R}^n | \dot{V}(x) = 0 }$
\end{itemize}
Wenn nur $x(t)=0$ in $\mathcal{S}$ bleiben kann, dann ist $x^* = 0$ global asymptotisch stabil


\subsection{Indirekte Methode von Lyapunov}

\subsubsection*{Zeitinvariante Systeme}
Linearisierung um Ruhelage $x^*$:\\
Systemmatrix $A = \left[ \frac{\partial \underline{f}(\underline{x})}{\partial \underline{x}} \right]_{x=x^*}$
\begin{itemize}
  \item $A$ ist negativ definit $\Rightarrow x^*$ ist lokal asymptotisch stabil
  \item $A$ ist indefinit oder positiv (semi-)definit $\Rightarrow x^*$ ist lokal instabil
  \item $A$ ist negativ semidefinit $\Rightarrow$ keine Aussage über $x^*$ möglich
\end{itemize}

\subsubsection*{Zeitvariante Systeme}
Linearisierung um Ruhelage $x^*$ \\
$\Rightarrow \dot{x} = A(t) x + f_1(x,t)$, wobei \\
\begin{itemize}
  \item $A(t) = \left[ \frac{\partial f(x,t)}{\partial x} \right]_{x=x^*}$
  \item $f_1(x,t)$ Restterm
\end{itemize}

Bedingung: Vereinfachte Linearisierung $\dot{x}  = A(t) x$ gültig falls: \\
$\lim\limits_{\|x\| \rightarrow 0} \sup\limits_{t \geq 0} \frac{\|f_1(x,t)\|}{\|x\|} = 0$

Stabilität des nichtlinearen Systems
\begin{itemize}
  \item $x^*$ ist uniform asymptotisch stabil in Linearisierung \\
    $\Rightarrow x^*$ ist uniform asymptotisch stabil im nichtlinearen System
  \item $x^*$ ist instabil in Linearisierung \\
    $\Rightarrow$ keine Aussage über $x^*$ im NL System möglich
  \item $x^*$ ist instabil in Linearisierung und $A(t) = A_0 = const$ \\
    $\Rightarrow x^*$ instabil im NL System
\end{itemize}

\subsubsection*{Stabilität von LTV Systemen(1)}
Ruhelage des LTV Systems ist exponentiell stabil wenn $\left[ A(t) + A(t)^T \right]$ negativ definit ist für alle $t$

\subsubsection*{Stabilität von LTV Systemen(1)}
Ruhelage des LTV Systems ist exponentiell stabil wenn $A(t)$ negativ definit ist und $A(t)$ beschränkt ist, dh \\
$\int\limits_{0}^{\infty}A(t)^T A(t) \diff t < \infty$

\subsection{Instabilität}
Falls Stabilität nicht nachgewiesen werden kann, versucht man Instabilität nachzuweisen

\subsubsection*{Satz von Chetaev}
\begin{itemize}
  \item $x^* = 0$ ist Ruhelage
  \item $V(x)$ ist stetig diff'bar, $V(0)=0, V(x_0)>0$ für $\|x_0\| > 0$
  \item $\mathcal{U} := \left\{ x \in \mathcal{B}_\varepsilon | V(x) > 0 \right\}$
\end{itemize}
Wenn $\dot{V}(x) > 0$ auf $\mathcal{U}$, dann ist $x^*=0$ instabil

Bemerkung:\\
\begin{itemize}
  \item $V(x)$ muss keine pdf sein
  \item Es genügt Menge $\mathcal{U}$ zu finden, so dass: $V(x) > 0$ und $0 \in \mathcal{U}$
\end{itemize}


\subsection{Einzugsgebiet}
Falls asymptotisch stabile Ruhelage nicht global asymptotisch stabil \\
$\Rightarrow$ Einzugsgebiet bestimmen, in der die Ruhelage lokal asymptotisch stabil ist

\subsubsection*{Einzugsgebiet, Domain of Attraction, Basin}
$\mathcal{A}(x^*) := \left\{ x_0 | \lim\limits_{t \rightarrow \infty} \Phi(t,t_0,x_0) = x^* \right\}$ \\
mit $\Phi(t,t_0,x_0)$ als Lösung der DGL

\subsubsection*{Bestimmen des Einzugsgebiets}
\begin{itemize}
  \item $x^*$ ist Ruhelage, asymptotisch stabil
  \item $\mathcal{V} = {x^*} \cup \left\{ x | V(x) > 0, \dot{V}(x) < 0 \right\}$
  \item $\mathcal{E}_c = \left\{ x | V(x) \leq c \right\}$
\end{itemize}
Wenn $\mathcal{E}_c \subseteq \mathcal{V}$ und $\mathcal{E}_c$ ist beschränkt, dann ist $\mathcal{E}_c$ Teilmenge des Einzuggebiets


\subsection{Lyapunov-basierter Reglerentwurf}
\begin{enumerate}
  \item $V(x)$ so aufstellen, dass $u$ in $V(x)$ und in $\dot{V}(x)$ vorkommt
  \item $u$ so einstellen, dass $V(x) > 0$ und $\dot{V}(x) < 0$
\end{enumerate}


\section{Passivität}
Achtung: $V(x)$ ist abstrakte Speicherfunktion \\
Energiespeicherfunktion zB aus physikalischer Energiebetrachtung\\

Verallgemeinerte Energiebilanz und Versorgungsrate eines Systems:\\
$\int\limits_{0}^{t} s(u,y) \diff \tau + V(x(0)) = \int\limits_{0}^{t} g(\tau) \diff \tau + V(x(t))$  \\
Mit: \\
Netto-Energiezufluss: $\int\limits_{0}^{t} s(u,y) \diff \tau$ \\
Versorgungsrate: $s(u,y)$ \\
Anfangs gespeicherte Energie: $V(x(0))$ \\
dissipierte Energie: $\int\limits_{0}^{t} g(\tau) \diff \tau$ \\
dissipierte Leistung: $g(\tau)$ \\
gespeicherte Energie: $V(x(t))$ \\
Es gilt $\int\limits_{0}^{t}|s(u(\tau), y(\tau))| \diff \tau < \infty$

\subsubsection*{Dissipativität (dissipativ bzgl $s(u,y)$)}
$V(x)$ ist psdf \\
Integrale Dissipativitätsungleichung: $\int\limits_{0}^{t} s(u,y) \diff \tau + V(x(0)) \geq V(x(t))$ \\
Differentielle Dissipativitätsungleichung: $s(u,y) \geq \dot{V}(x(t))$

\subsubsection*{Passivität}
Dissipativ bzgl spezieller Versorgungsrate $s(u,y) = y^T u$ \\
$V(x)$ ist psdf \\
Integrale Passivitätsungleichung: $\int\limits_{0}^{t} y^T u \diff \tau + V(x(0)) \geq V(x(t))$ \\
Differentielle Passivitätsungleichung: $s(u,y) \geq \dot{V}(x(t))$ \\
\begin{tabular}{cccc}
  streng passiv: &  $\Rightarrow$ bei '$>$' & bzw $g(t) > 0$ \\
  verlustlos:    &  $\Rightarrow$ bei '=' & bzw $g(t) = 0$
\end{tabular}

\subsection{Passivität und Stabilitätseigenschaften}

\subsubsection*{Passivität und Lyapunov-Stabilität}
\begin{itemize}
  \item System ist passiv
  \item $V$ ist stetig diff'bar und psd
\end{itemize}
$\Rightarrow$ Ruhelage $x=0$ ist stabil iSvL

\subsubsection*{Null-Zustandsbeobachtbarkeit}
Nur $x^*=0$ kann in $\mathcal{S} = \left\{ x \in \mathbb{R} | h(x,0)=0 \right\}$ bleiben

\subsubsection*{Passivität und asymptotische Stabilität}
$x^*=0$ ist asymptotisch stabil wenn eine der beiden Punkte zutrifft: \\
\begin{itemize}
  \item System ist streng passiv
  \item über $V(x)$:
    \begin{itemize}
      \item System ist passiv
      \item $V(x)$ ist stetig diff'bar und pdf
      \item $\dot{V}(x) = 0 \Leftrightarrow y = 0$
      \item Null-Zustand beobachtbar
    \end{itemize}
\end{itemize}

Wenn $V(x)$ zusätzlich radial unbeschränkt ist $\Rightarrow x^* = 0$ ist global asymptotisch stabil


\section{Passivitätsbasierte Regelung}
$x^* = 0$ ist global asymptotisch stabil\\
$\Rightarrow$ System kann stabilisiert werden mit $u = -\Phi(y)$, wobei:
\begin{itemize}
  \item $\Phi$ ist lokal Lipschitz-stetig
  \item $\Phi$ ist beliebig
  \item $\Phi(0) = 0$
  \item $y^T \Phi > 0$ für $y \neq 0$
\end{itemize}

\begin{tabular}{cccc}
mögliche $\Phi$:  & $\Phi = k_i \text{sat}(y_i)$ & \\
                  & $\Phi = \frac{2 k_i}{\pi}\text{atan}(y_i)$ &
\end{tabular}

\subsubsection*{Feedback-Passivierung}
Ziel: Nicht-Passive Systeme in passive transformieren durch spezielle Wahl der Ausgangsfunktion $y=h(x)$ \\
$\dot{x} = f(x) + G(x)u$ \\
$\Rightarrow$ Ausgang $y = h(x) \overset{\text{def}}{=} \left[ \frac{\partial V}{\partial x} G \right]^T$ \\
Ist Ausgang dann Null-Zustandsbeobachtbar $\Rightarrow$ es kann global stabilisierendes Regelgesetz gefunden werden


\section{Feedback-Linearisierung}
Nichtlineare System-Transformation: $z = \varphi (x)$

\subsection{Vorgehen}
\begin{enumerate}
  \item Zustandstransformation: $z = \varphi(x)$
  \item NL-RNF aufstellen
  \item Überprüfen ob $\varphi(x)$ ein Diffeomorphismus ist
  \item Feedback-linearisierendes Regelgesetz aufstellen
\end{enumerate}

\subsubsection*{Nichtlineare Regelungsnormalform, NL-RNF}
\begin{align*}
  \dot{z}_1 &= z_2 \\
  \dot{z}_2 &= z_3 \\
            &\dots \\
  \dot{z}_n &= a(x) + b(x) u \\
\end{align*}

\subsubsection*{Diffeomorphismus}
$z = \varphi(x)$ ist (lokal) gültige Zustandstransformation wenn $\nabla \underline{\varphi}$ nicht singulär ist, $\Leftrightarrow \det(\nabla \varphi) \neq 0$ \\
$\nabla \underline{\varphi} = \left[ \frac{\partial \varphi_i}{\partial x_j} \right]$, Jacobi-Matrix

\subsubsection*{Feedback-linearisierendes Regelgesetz}
$u(x) = \frac{1}{b(x)}[v - a(x)]$ \\
$\Rightarrow \dot{z}_n = v $


\section{E/A-Linearisierung}

\subsubsection*{Vorgehen}
\begin{enumerate}
  \item Ausgang $y$ festlegen, dessen dynamische Antwort auf Reglereingang $v$ linearisiert werden soll
  \item Zeitliche Ableitung des Ausgangs $y$ liefert nach einigen Schritten  die E/A-Beziehung in RNF
  \item Aus RNF das feedback-linearisierende Regelgesetz aufstellen
  \item Bei Bedarf Systemtransformation durchführen, so dass $\dot{z}_n = v$
\end{enumerate}

$\dot{x} = f(x) + g(x)u$ \\
$\Rightarrow \dot{y}(x) = \frac{\partial h}{\partial x}f(x) + \frac{\partial h}{\partial x}g(x)u = L_f h(x) + L_g h(x) u$ \\

\subsubsection*{zu 2.}
$y$ so lange ableiten bis: $\overset{(r)}{y} = a(x) + b(x) u$ \\
\begin{align*}
  \dot{y}           &= L_f h                            &(\text{mit} L_g h(x) = 0) \\
  \ddot{y}          &= L_f^2 h                          &(\text{mit} L_g L_f h(x) = 0) \\
                    &\dots                              & \\
  \overset{(r)}{y}  &= L_f^r h + L_g L_f^{r-1} h(x) u   &
\end{align*}

\subsubsection*{zu 3.}
$u(x) \overset{!}{=} \frac{1}{b(x)} [v - a(x)]$ \\
Neuer virtueller Systemeingang: $v = \overset{r}{y}$ \\
Regelgesetz: $u(x) = \frac{v - L_f^r h(x)}{L_g L_f^{r-1} h(x)}$

\subsubsection*{Relativer Grad bzw Differenzengrad}
Vollstandige Linearisierung: $r = n$ \\
interne Dynamik vorhanden: $r < n$ \\
Nulldynamik: $y(t) = 0, \forall t$, mit interner Dynamik


\subsection{Zustands-Linearisierung}
$\dot{x} = f(x) + g(x)u$ \\
$\dot{z} = \nabla \varphi(x) \left( f(x) + g(x)u \right)$

\subsubsection*{Vorgehen}
\begin{enumerate}
  \item Nichtlineare Zustandstransformation bestimmen $\Rightarrow \varphi(x)$
  \item Regelgesetz bestimmtn
\end{enumerate}

\subsubsection*{zu 1.}
GLS lösen: \\
\begin{align*}
  \underbrace{
  \begin{bmatrix}
    g^T \\
    [\text{ad}_f g]^T \\
    \vdots \\
    [\text{ad}^{n-2}_f g]^T \\
    [\text{ad}^{n-1}_f g]^T \\
  \end{bmatrix}
  }_{S^T}
  \left[ \frac{\partial \varphi_1(x)}{\partial x} \right]^T
  =
  \begin{bmatrix}
    0 \\
    0 \\
    \vdots \\
    0 \\
    g^*
  \end{bmatrix}
\end{align*}

Matrix $S$ ist Erreichbarkeitsmatrix \\

GLS ist gleichbedeutend mit: \\
$ L_g L_f^i \varphi_1(x) =
\begin{cases}
  0, & i = 0, \dots, n-2 \\
  \hat{g}^*(x), & i = n-1
\end{cases}$

ist gleichbedeutend mit: \\
$\left[ \frac{\partial \varphi_1(x)}{\partial x} \right] \text{ad}_f^i g(x) = 
\begin{cases}
  0, & i = 0, \dots, n-2 \\
  \hat{g}(x), & i = n-1
\end{cases}$

wobei:\\
$\hat{g}^*, g^* \neq 0$ \\
$g* = (-1)^{n-1} \hat{g}^*$\\

Dann nach $\frac{\partial \varphi_1}{\partial x}$ auflösen und daraus $\varphi_1$ bestimmen. \\
Für die restlichen $\varphi_i$ gilt: $\varphi_i(x) = L_f^i \varphi_1$

\subsubsection*{zu 2.}
Regelgesetz: $u(x) = \frac{1}{L_g L_f^{n-1} \varphi_1(x)} \left( v - L_f^n \varphi_1(x) \right)$\\
wobei $v$: neuer Regeleingang



\section{Flachheitsbasierte Regelung}

\subsubsection*{Vorgehen}
\begin{enumerate}
  \item Flachheitsanalyse
  \item Flachheitsbasierte Steuerung
  \item Flachheitsbasierte Folgeregelung
\end{enumerate}

\subsubsection*{zu 1. Flachheitsanalyse}

System ist flach wenn folgende Bedingungen erfüllt sind:
\begin{itemize}
  \item es gibt (fiktiven) Ausgang $y = \Phi(x, u, \dot{u}, \dots, \os{(\alpha)}{u})$ \\
    mit dim $y$ = dim $u$
  \item eine (lokal) eindeutige Zustandsfunktion kann gefunden werden: \\
    $x = \Psi_1 (y, \dot{y}, \dots, \os{(\gamma)}{y})$
  \item eine (lokal) eindeutige Eingangsfunktion kann gefunden werden: \\
    $u = \Psi_2 (y, \dot{y}, \dots, \os{(\gamma +1)}{y})$
\end{itemize}

\subsubsection*{Flachen Ausgang bestimmen}
\begin{itemize}
  \item Ausgang sollte möglichst viel Information über das dynamische Systemverhalten haben
  \item Sukzessive zeitliche Ableitung des Kandidaten zur Herleitung von Gleichungen zur Bestimmung von $x$ und $u$
  \item $y$ muss so oft abgeleitet werden, bis aus dem resultierenden GLS von $y,\dots,\os{\gamma}{y}$ alle unbekannten $x$ und $u$ (lokal) bestimmt werden können
  \item Kandidat ist umso erfolgversprechender, je häufiger abgeleitet werden kann ohne dass Eingänge $u$ auftauchen
\end{itemize}

Danach $x = \Psi_1 (y, \dot{y}, \dots, \os{(\gamma)}{y})$ und $u = \Psi_2 (y, \dot{y}, \dots, \os{(\gamma +1)}{y})$ bestimmen

\subsubsection*{zu 2. Flachheitsbasierte Steuerung}

Solltrajektorie bestimmen:
\begin{enumerate}
  \item $y_d$ bestimmen: entweder vorgegeben oder\\
    falls $y_d$ nicht vorgegeben, dann aus $x_d$ oder Regelgröße $w$ bestimmen
  \item zugehörige $x_d$ und $u_d$ bestimmen
\end{enumerate}

\subsubsection*{zu 3. Flachheitsbasierte Folgeregelung}
Zustandsrückführung und Nichtlineares Regelgesetz aufstellen
\begin{enumerate}
  \item fiktive (differentierte) Ausgänge $\left[ y, \dots, \os{(\alpha)}{y} \right]$ als Eingänge $v$ einführen
  \item Nichtlineares Regelgesetz aufstellen: $u = \Psi\left( y, \dots, \os{(\alpha)}{y}, v \right)$
  \item Zustandstransformation: $z = \dots$
  \item Zustands-DGL: $\dot{z} = \dots$
\end{enumerate}


\section{Backstepping}

\subsection{Anwendungsgebiet}
$u \rightarrow \dot{x}_n \rightarrow \int \dots \rightarrow \dot{x}_i \rightarrow \int \rightarrow \dot{x}_1 \rightarrow \int \rightarrow x_1$

\begin{align*}
  \dot{x}_1 &=      & f_1(x_1) + g_1(x_1)x_2 \\
  \dot{x}_2 &=      & f_2(x_1, x_2) + g_2(x_1, x_2)x_3 \\
            &\vdots & \\
  \dot{x}_i &=      & f_i(x_1, \dots, x_i) + g_i(x_1, \dots, x_i)x_{i+1} \\
            &\vdots & \\
  \dot{x}_n &=      & f_n(x_1, \dots, x_n) + g_n(x_1, \dots, x_n)u
\end{align*}


\subsection{Verfahren (rekursiv anwenden)}

System wird in Teilsysteme unterteil. Ausgang des einen Teilsystems ist Pseude-Stellgröße des nachfolgenen Systems.

\begin{enumerate}
  \item Transformiertes Teilsystem aufstellen\\
    $z = \dots$\\
    $\dot{z} = \dots$
  \item Pseudo-Stellgröße festlegen
  \item Partielle Lyapunov Funktion aufstellen:
    \begin{itemize}
      \item  Meist:\\
        $V_1 = \frac{1}{2} z_1^2$\\
        $V_i = V_{i-1} + \frac{1}{2}z_i^2$ \\
        $V_n = \frac{1}{2} \sum\limits_{i=1}^{n} z_i^2$
      \item $\dot{V}_i = \Psi(z, x_{i+1}) \Rightarrow x_{i+1}$ so festlegen, dass $\dot{V}_i^* < 0$
    \end{itemize}
  \item Funktion für gewünschte Stellgröße $\alpha_i$ bestimmen: $x_{i+1} := \alpha_i$
  \item So lange rekursiv anwenden bis $\alpha_i = u$
\end{enumerate}



\section{Sliding Mode Regelung}

System: $\dot{x} = f(x) + g(x)u + d(t)$ \\
wobei $d(t)$ unbekannte Störfunktion ist \\
Schaltmannigfaltigkeit: $S = \left\{ x \in \mathbb{R}^n | s(x) = 0 \right\}$ \\
unstetige Stellgröße:
$u(x) =
\begin{cases}
  u^+(x) & \text{für} s(x) > 0 \\
  u^-(x) & \text{für} s(x) < 0 \\
\end{cases}$\\
unstetiges Systemverhalten:
$\dot{x} =
\begin{cases}
  f^+(x) & \text{für} s(x) > 0 \\
  f^-(x) & \text{für} s(x) < 0 \\
\end{cases}$\\
Regelziel: Systemzustand soll nach ersten Kontakt auf Schaltmannigfaltigkeit $s(x) = 0$ bleiben\\
Gezielte Unterdrückung von Störung ist möglich wenn:
\begin{itemize}
  \item $d(x,t)$ liegt in dem von $g(x)$ aufgespannten Raum
  \item $|d_i| < D_i, D_i = const \in \mathbb{R}$
\end{itemize}

\subsubsection*{Vorgehen}
\begin{enumerate}
  \item Diskontinuierliche Reglerfunktion finden, so dass System in endlicher Zeit in den Sliding Mode geht
  \item Schaltmannigfaltigkeit so wählen, dass im Sliding Mode gewünschte Systemdynamik auftritt
\end{enumerate}

\subsubsection*{zu 1.}
Um in den Sliding Mode zu kommen muss gelten: 
\begin{itemize}
  \item $s_i \dot{s}_i < 0$
  \item $\lim\limits_{s_i(x) \rightarrow 0^+} \dot{s}_i(x) = k^- < 0$ und $\lim\limits_{s_i(x) \rightarrow 0^-} \dot{s}_i(x) = k^+ > 0$
\end{itemize}

\subsubsection*{zu 2., Beschreibung des Systemverhaltens $\dot{x}$}

\subsubsection*{Idealer Sliding Mode nach Filippov}
$\dot{x} = f(x)$\\
Ansatz: $\dot{x}_\text{fi} = \alpha f^+(x) + (1-\alpha) f^-(x)$ mit $0 \leq \alpha \leq 1$ \\
Bedingung: $\dot{s}(x_\text{fi}) = \frac{\partial s}{\partial x} \dot{x}_\text{fi} = 0$ \\
Man erhält: $\alpha = \frac{ \frac{\partial s}{\partial x} f^-(x)  }{ \frac{\partial s}{\partial x} (f^-(x) - f^+(x))  }$ \\
und somit:\\ $\dot{x}_\text{fi} = \frac{ \frac{\partial s}{\partial x} f^-(x)  }{ \frac{\partial s}{\partial x} (f^-(x) - f^+(x)) } f^+(x) - 
\frac{ \frac{\partial s}{\partial x} f^+(x)  }{ \frac{\partial s}{\partial x} (f^-(x) - f^+(x)) } f^-(x)$ \\
Wobei: $\frac{\partial s}{\partial x}f^- \geq 0$ und $\frac{\partial s}{\partial x}f^+ \leq 0$ \\

\subsubsection*{Idealer Sliding Mode nach der Equivalent Control Method}
$\dot{x} = f(x) + g(x)u$ \\
Es gilt: $s(x) = 0, \dot{s}(x) = 0$ \\
Daraus folgt: $\dot{s}(x) = L_f s(x) + L_g s(x) \tilde{u}_\text{eq}$ \\
Kontinuierliche Ersatzstellgröße: $\tilde{u}_\text{eq} = -L_g s(x) ^{-1} L_f s(x)$ \\
Systemdynamik: $\dot{x} = f(x) - g(x) L_g s(x)^{-1} L_f s(x)$

































% Ende des Inhalts
\end{multicols*}
% Ende der Spalten


% Dokumentende
% ======================================================================
\end{document}
